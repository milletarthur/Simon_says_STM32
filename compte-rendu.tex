\documentclass[10pt,a4,french]{article}
\usepackage{enumitem}
\usepackage{listings}
\usepackage[margin=3cm]{geometry}

\title{Compte-Rendu TP SPC}
\author{Arthur MILLET\\ Rémi GAULMIN}
\date{19 Mars 2024}

\begin{document}

\lstset{language=C, frame=single}

\maketitle

\section{Description du projet}

Notre projet est le jeu électronique du \textbf{Simon}.
\\

Simon va éclairer une des quatre couleurs :
le joueur doit alors appuyer sur la touche de la couleur qui vient de s'allumer.
Ensuite, le jeu répète la même couleur, puis ajoute au hasard une nouvelle couleur.
Le joueur doit reproduire cette nouvelle séquence.
Chaque fois que le joueur reproduit correctement la séquence, le jeu ajoute une nouvelle couleur.
\\

Pour sélectionner la difficulté, le joueur devra rentrer manuellement dans la ligne série le nombre d'essais (positif et inférieur à 256) lorqu'il y sera invité.

Le joueur doit utiliser le potentiomètre pour parcourir les couleurs,
puis sélectionner la couleur avec le bouton de la carte fille.

\section{Le rôle des interruptions}

Le rôle des interruptions dans notre projet est assez minime puisque nous ne nous servons que d'un timer.
En effet, le jeu du Simon affiche une séquence puis attend que le joueur fasse sa séquence : tout peut être géré par la fonction \texttt{main}.
Pour afficher une séquence, il faut un espacement de temps,
c'est pourquoi nous avons tout de même besoin d'une interruption qui incrémente la variable globale \texttt{temps\_globale}.

\section{Les principaux schémas algorithmiques}

\section{Description du travail réalisé}

Ce qui a marché :

\begin{itemize}
\end{itemize}

Ce qui n'a pas marché :

\begin{itemize}
	\item Nous avons voulu rajouter un buzzer pour associer une sonorité à chaque couleur et nous avions réussi,
		malheureusement la carte \texttt{STM32} ne fournissait pas assez d'intensité pour les LEDs, le buzzer, le potentiomètre et le bouton.
\end{itemize}

\section{Points importants}

\section{Pour aller plus loin}

\end{document}
